%\documentclass[a4paper,10pt]{article}
\documentclass[a4paper,10pt]{article}
\usepackage[margin=1.5in]{geometry}
\usepackage{graphicx}
\usepackage{enumitem}
\usepackage{listings}
\usepackage{color}
\usepackage[utf8]{inputenc}
\lstdefinestyle{customc}{
  belowcaptionskip=1\baselineskip,
  breaklines=true,
  xleftmargin=\parindent,
  language=C,
  showstringspaces=false,
  basicstyle=\footnotesize\ttfamily,
  keywordstyle=\bfseries\color{green},
  commentstyle=\itshape\color{black},
  identifierstyle=\color{blue},
  stringstyle=\color{red},
}
\lstset{escapechar=@,style=customc}
\title{Mouse running task}
\author{}
\date{}

\pdfinfo{%
  /Title    ()
  /Author   ()
  /Creator  ()
  /Producer ()
  /Subject  ()
  /Keywords ()
}

\begin{document}
\maketitle
\section{Behavioral task}
You will setup a simple behavioral task with a head-fixed mouse. The mouse will run on a styrofoam wheel, 
while you acquire his running speed / position with a rotary encoder, reward him with water using a valve for 
reaching a target speed and deliver auditory feedback. The behavior will be controlled entirely
by an Arduino microcontroller. MATLAB will be used only to record and display the data.

You will have to decide on a range of speeds to reward. Perhaps it's best to start with a wide range
and narrow it down as the mouse becomes comfortable with the task. To measure running, you will be using a rotary encoder
 (more on that later), which produces 1000 clicks per revolution. One revolution of the wheel per second 
 is pretty pretty fast running speed and that can be the eventual goal to aim for. You will also 
 have to pick a minimum time, for which the mouse has to maintain target speed to receive reward, 
and a transfer function to convert running speed into auditory feedback frequency.

\subsection{Behavior II: Implementation day}
\begin{itemize}
 \item Setup running wheel, rotary encoder, piezo buzzer and valve
 \item Make the circuit for controlling piezo and valve with arduino
 \item Write arduino code to track speed and deliver reward and feedback
 \item Write MATLAB code to acquirespeed signals and reward timestamps sent over serial port
 and plot them in real time
\end{itemize}

\subsection{Behavior III: Behavioral session}
\begin{itemize}
 \item Train a mouse to reach a target position by running on the wheel - auditory feedback based on how similar
 current speed is to target
 \item Gradually narrow down the target speed window and increase time mouse has to maintain it
\end{itemize}
\pagebreak
\section{Rotary encoders}
\begin{figure}[here]
\includegraphics[width=0.9\textwidth]{encoder.png}
\caption{Rotary encoder waveform}
\end{figure}

You will use a rotary encoder to read out the rotation of the wheel. It is a quadrature encoder has 
two output lines, which turn on and off as the shaft rotates. Depending on the direction
of rotation, one or the other line will flip on first (see figure). 

To read the encoder you will need to use the following lines:

\begin{description}[align=right]
\item [BROWN] 5V power
\item [WHITE] GND
\item [GRAY] Line A
\item [GREEN] Line B
\end{description}

The encoders you are using produce 1000 pulses per revolution and are best read out by attaching
an interrupt to one of the output lines. Note that not all digital lines can be used to trigger
interrupts. The line below will add an interrupt to pin number INPUTPIN that will trigger
when that pin goes on or off.

\begin{lstlisting}
attachInterrupt(digitalPinToInterrupt(INPUTPIN), encoderFunc, CHANGE);
\end{lstlisting}

\section{Timers}
Timers are the Arduino way of making sure that code (such as to playing a cycle of a sound 
waveform or measuring animal running speed) gets executed at specific moments in time (duh!).
They are configured in a somewhat arcane way using the timer registers. Arduino Uno has 3 timers,
some of which are used by built-in Arduino routines.

\begin{description}[align=right]
\item [Timer0] 8 bit, used by \textit{delay()}, \textit{millis()} and \textit{micros(s)}
\item [Timer1] 16 bit
\item [Timer2] 8 bit, used by \textit{tone()}
\end{description}

Two common ways to 
use them is with compare match and overflow interrupts. The former triggers when the timer
reaches a pre-defined value, while the latter when it overflows 255 (for an 8 bit timer) or 
65536 (16 bit). The overflow interrupt can be used to trigger events at specific intervals
by pre-setting the timer value (see example below).

\begin{lstlisting}
void setup() {
  noInterrupts();           // disable all interrupts
  // clear timer1 control registers
  TCCR1A = 0;
  TCCR1B = 0;
  // this sets CS12 bit of the control register to 1
  TCCR1B |= (1 << CS12);    // 256 prescaler 
  // set timer value to overflow at 100 Hz
  TCNT1 = 64911;   // 65536-16MHz/256/100
  // set Time Overflow Interrupt Enable bit to 1
  TIMSK1 |= (1 << TOIE1);   // enable timer overflow interrupt
  interrupts();  
}

ISR(TIMER1_OVF_vect)        // interrupt service routine 
{
  TCNT1 = 64911;  // reset the timer
  // do stuff here
}
\end{lstlisting}

\textbf{CS12} and \textbf{TOIE1} are keywords that store the position of the relevant bits in the 
in the timer control registers. You can find their full description online. We use 
the bitwise OR operator |= and bit-shift operator << to turn on specific bits in 
the register.

\section{Serial communication}
You will save and analyze running speed data acquired by the Arduino in MATLAB.
The following code sample configures a serial object to read Arduino
output.

\begin{lstlisting}[language=matlab]
 % initialize serial communication
s = serial('COM6');
set(s,'BaudRate', 9600);
set(s,'DataBits', 8);
set(s,'StopBits', 1);
fopen(s);
s.ReadAsyncMode = 'continuous';
readasync(s);
\end{lstlisting}

You can then read from the serial port as you would from a file.
\end{document}
